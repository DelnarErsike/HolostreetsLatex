% commands.tex - for commands we expect the user to use
% executed after definitions but before configuration

\newcommand{\TitleImage}[1]{\def\titleimage{#1}}

\newcommand{\shadowrunlegaltext}{Shadowrun, Sixth World, and Matrix, and associated graphics and logos are registered trademarks and/or trademarks of The Topps Company, Inc., in the United States and/or other countries. Catalyst Game Labs and the Catalyst Game Labs logo are trademarks of InMediaRes Productions, LLC.}

\newcommand{\beginnumbering}
{
    \InsertMark{drawPageNumber}{1}
    \setcounter{page}{1}
}

\newenvironment{chatter}
{
    \tt
}
{
}

\DeclareRobustCommand{\tableheader}[1]{
    \textcolor{corecolor}{\headerfont\MakeUppercase{#1}}
}
   
\newcommand{\quality}[3]{
    \subsection{#1}
    \begin{bfseries}
        #2
    \end{bfseries}
    
    #3
}

\newcommand{\adeptpower}[3]{
    \subsection{#1}
    \begin{FlushLeft}
    \begin{bfseries}
        #2
    \end{bfseries}
    \end{FlushLeft}

    #3
}

%- 1st: name
%- 2nd: type line
%- 3rd: type
%- 4th: range
%- 5th: damage (optional kind of?)
%- 6th: duration
%- 7th: drain
%- 8th - content
\newcommand{\combatspell}[8]{
    \undef\@headerrule
    \subsection{#1}
    \vspace{-\@postheaderlength}
    \def\@headerrule{}
    \begin{FlushLeft}
        \ifstrempty{#2}{}{{\small\headerfont\MakeUppercase{#2}}}
        \vspace{2pt plus 2pt}
        \hrule
        \vspace{2pt plus 2pt minus 2pt}
        \begin{tabularx}{\columnwidth}{XXX}
            \textbf{Type:} #3     & \textbf{Range:} #4  & \ifstrempty{#5}{}{\textbf{Damage:} #5} \\
            \textbf{Duration:} #6 & \textbf{Drain:} #7 &           \\
        \end{tabularx}
    \end{FlushLeft}
    
    #8
}

%- 1st: name
%- 2nd: type line
%- 3rd: type
%- 4th: range
%- 5th: damage (optional kind of?)
%- 6th: duration
%- 7th: drain
%- 8th - content
\newcommand{\spell}[7]{
    \combatspell{#1}{#2}{#3}{#4}{}{#5}{#6}{#7}
}

\newcommand{\yen}{¥}

\newcommand{\todo}[1]{{\color{red}\textbf{TODO: #1}}}

\newcommand{\matrixaction}[3]
{
    \subsubsection{#1}
    \toks0={#1}%
    \edef\param{\the\toks0}
    \begin{bfseries}
        \raggedright#2
    \end{bfseries}

    #3
}

\newenvironment{srsmalltable}[2][h]
{
    \begin{table}[#1]
        \begin{mdframed}[backgroundcolor=tablebackground, hidealllines=true, innertopmargin=12pt, innerbottommargin=12pt]
            \begin{small}
            \tableheader{\raggedright\Large #2}
            
            \tablefont
            \medskip
            \small
            \rowcolors{1}{tablebackground2}{tablebackground}
}
{
            \end{small}
        \end{mdframed}
    \end{table}
}

\newenvironment{srbigtable}[2][h]
{
    \begin{table*}[#1]
        \begin{mdframed}[backgroundcolor=tablebackground, hidealllines=true, innertopmargin=12pt, innerbottommargin=12pt]
            \begin{small}
            \tableheader{\raggedright\Large #2}
            
            \tablefont
            \medskip
            \small
            \rowcolors{1}{tablebackground2}{tablebackground}
}
{
            \end{small}
        \end{mdframed}
    \end{table*}
}

\newenvironment{srsmallbox}[2][h]
{
    \edef\@restorevalues{
    \parindent=\the\parindent
    \parskip=\the\parskip
    }
    \begin{figure}[#1]
        \begin{mdframed}[backgroundcolor=tablebackground, hidealllines=true, innertopmargin=12pt, innerbottommargin=12pt]
            \begin{small}
            \@restorevalues
            \tableheader{\raggedright\Large #2}
            
            \tablefont
            \medskip
            \rowcolors{1}{tablebackground2}{tablebackground}
}
{
            \end{small}
        \end{mdframed}
    \end{figure}
}

\newenvironment{srbigbox}[2][h]
{
    \edef\@restorevalues{
    \parindent=\the\parindent
    \parskip=\the\parskip
    }
    \begin{figure*}[#1]
        \begin{mdframed}[backgroundcolor=tablebackground, hidealllines=true, innertopmargin=12pt, innerbottommargin=12pt]
            \begin{small}
            \@restorevalues
            \tableheader{\raggedright\Large #2}
            
            \tablefont
            \medskip
            \small
            \rowcolors{1}{tablebackground2}{tablebackground}
}
{
            \end{small}
        \end{mdframed}
    \end{figure*}
}


\newdimen\scratch@dimen
% helper for multicolumn
% first arg: total number of columns
% second arg: content
% handles calculating the width
\newcommand{\wholerow}[2]
{
    \multicolumn{#1}{>{\scratch@dimen=\dimexpr\textwidth-2\tabcolsep\relax\hsize=\scratch@dimen}X}{#2}
}

% first - name
% second - how many arguments
% third - tabular column specifier
% fourth - first row (for column names)
\newcommand{\newcollection}[4]
{
    \makeatletter

    \immediate\write\@auxout{\detokenize{\@writefile{table_#1}{\begin{tabularx}{\textwidth}{#3}}}}
    \immediate\write\@auxout{\noexpand\@writefile{table_#1}{#4 \\}}
    \AtEndDocument{\immediate\write\@auxout{\detokenize{\@writefile{table_#1}{\end{tabularx}}}}}

    \expandafter\newcommand\csname #1\endcsname[3]
    {
        \write\@auxout{\noexpand\@writefile{table_#1}{##1 & ##2 & ##3 \\}}
    }

    \expandafter\newcommand\csname table#1\endcsname
    {
        \@starttoc{table_#1}
    }
    \makeatother
}
\newcommand{\srsixstats}[5]
{
    \begin{center}
    \begin{tabular}{ccccc}
        \rowcolor{tablebackground2}
        \tableheader{DR} & \tableheader{I/ID} & \tableheader{AC} & \tableheader{CM} & \tableheader{MOVE} \\
        \rowcolor{tablebackground}
        #1 & #2 & #3 & #4 & #5
    \end{tabular}
    \end{center}
    \par
}

\newcommand{\statline}[9]
{
    \begingroup
    \setlength{\tabcolsep}{0pt}
    \begin{tabularx}{\textwidth}{YYYYYYYYY}
        \rowcolor{tablebackground2}
        \tableheader{B} & \tableheader{A} & \tableheader{R} & \tableheader{S} & \tableheader{W} & \tableheader{L} & \tableheader{I} & \tableheader{C} & \tableheader{ESS} \\
        \rowcolor{tablebackground}
        #1 & #2 & #3 & #4 & #5 & #6 & #7 & #8 & #9 \\
    \end{tabularx}
    \endgroup
    \par
}

\newcommand{\@relay@statlineSpecial}[9]
{
    \begingroup
    \setlength{\tabcolsep}{0pt}
    \begin{tabularx}{\textwidth}{YYYYYYYYYY}
        \rowcolor{tablebackground2}
        \tableheader{B} & \tableheader{A} & \tableheader{R} & \tableheader{S} & \tableheader{W} & \tableheader{L} & \tableheader{I} & \tableheader{C} & \tableheader{ESS} & \tableheader{#8}\\
        \rowcolor{tablebackground}
        \@tmp & \@tmptwo & #1 & #2 & #3 & #4 & #5 & #6 & #7 & #9 \\
    \end{tabularx}
    \endgroup
    \par
}

\newcommand{\statlineSpecial}[2]
{
    \def\@tmp{#1}
    \def\@tmptwo{#2}
    \@relay@statlineSpecial
}

\newcommand{\@relay@statlineDoubleSpecial}[9]
{
    \begingroup
    \setlength{\tabcolsep}{0pt}
    \begin{tabularx}{\textwidth}{YYYYYYYYYYY}
        \rowcolor{tablebackground2}
        \tableheader{B} & \tableheader{A} & \tableheader{R} & \tableheader{S} & \tableheader{W} & \tableheader{L} & \tableheader{I} & \tableheader{C} & \tableheader{ESS} & \tableheader{#6} & \tableheader{#8}\\
        \rowcolor{tablebackground}
        \@tmp & \@tmptwo & \@tmpthree & \@tmpfour & #1 & #2 & #3 & #4 & #5 & #7 & #9 \\
    \end{tabularx}
    \endgroup
    \par
}

\newcommand{\statlineDoubleSpecial}[4]
{
    \def\@tmp{#1}
    \def\@tmptwo{#2}
    \def\@tmpthree{#3}
    \def\@tmpfour{#4}
    \@relay@statlineDoubleSpecial
}

\mdfdefinestyle{statblock@srfive}{backgroundcolor=tablebackground, hidealllines=true, innertopmargin=0pt, innerbottommargin=0pt, innerleftmargin=0pt, innerrightmargin=0pt, nobreak=true}
\mdfdefinestyle{statblock@srsix}{backgroundcolor=tablebackground, hidealllines=true, innertopmargin=8pt, innerbottommargin=8pt, innerleftmargin=8pt, innerrightmargin=8pt, nobreak=true}

\newenvironment{statblock@internal}
{
    \ifthenelse{\isundefined{\@nomargin}}
    {
        \gdef\mdframedopt{statblock@srsix}
    }
    {
        \gdef\mdframedopt{statblock@srfive}
    }
    \begin{mdframed}[style=\mdframedopt]
        \setlength\parindent{0pt}
        \tablefont
        \footnotesize
        
        \begin{flushleft}
}
{
        \end{flushleft}
    \end{mdframed}
}

\newenvironment{statblock}
{
    \begin{statblock@internal}
}
{
    \end{statblock@internal}
}

\newenvironment{statblock*}
{
    \begingroup
    \def\@nomargin{}
    \begin{statblock@internal}
}
{
    \end{statblock@internal}
    \endgroup
}

\newcommand{\statblockHeader}[1]
{
    \tableheader{\normalsize #1}\par\vspace{5pt}
}

\newenvironment{sr5properties}
{
    \begingroup
    \newcommand{\property}[2]{\cellcolor{tablebackground2}\textbf{##1} & \multicolumn{2}{>{\hsize=2\hsize}X}{\cellcolor{tablebackground}##2} \\}
    \tabularx{\textwidth}{XXX}
}
{
    \endtabularx
    \endgroup
}

\newenvironment{sr6properties}
{
    \begingroup
    \newcommand{\property}[2]{\textbf{##1:} ##2\par}
    \newenvironment{weaponlist}
    {
        \begingroup
        \textbf{Weapons:}
        \begin{list}{}
        {
            \setlength\topsep{0pt}
            \setlength\itemindent{-6mm}
            \setlength\leftmargin{12mm}
            \setlength\itemsep{0pt}
        }
    }
    {
        \end{list}
        \endgroup
    }
}
{
    \endgroup
}

